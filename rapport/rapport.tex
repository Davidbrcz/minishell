\documentclass[fr]{article}
%\usepackage{url}
\usepackage[francais]{babel}
\usepackage[utf8]{inputenc}
\usepackage[T1]{fontenc}
\usepackage{fancyvrb}
\usepackage{times}
\usepackage{color}
\usepackage[colorlinks=true,linkcolor=blue,urlcolor=blue,filecolor=blue]{hyperref}
\usepackage{courier}
\usepackage{listings}
\usepackage{float}
\usepackage{enumitem}
\usepackage[table]{xcolor}
\lstset{
  language=C++,
  captionpos=b,
  tabsize=8,
  frame=lines,
  keywordstyle=\color{blue},
  numbers=none,
  numberstyle=\tiny,
  numbersep=5pt,
  breaklines=true,
  showstringspaces=false,
  basicstyle=\small\ttfamily
}

\definecolor{webred}{rgb}{0.5,0,0}
\usepackage{makeidx}
\makeindex
\usepackage{tabularx}
\usepackage{pgf}

\begin{document}
\date{\today}

\title{Rapport sur la réalisation de minishell}
\author{Quentin Deroubaix, Côme David}
\maketitle
%\tableofcontents

\section{Cahier des charges et réalisation}
Le but du mini projet était de se familiariser avec la gestion des
processus et file descriptors dans un environement type Linux. Pour ce
faire, il fallait réaliser un mini-shell qui gérait au moins les pipes
et les redirections.


A l'heure actuelle, notre projet implémente :

\begin{itemize}
\item Une gestion des pipes  
\item redirection >, $\gg$ et <
\item gestion du changement de dossier courant
\item La vérification de la cohérence de la séquence de commandes
  entrée par l'utilisateur
\end{itemize}

\section{Gestion des tableau et des chaines de caractères}
Afin de simplifier l'écriture du programme, une petite bibliothèque de
gestion des chaînes de caractères ainsi que des tableaux
de taille dynamique de type aribitraire a été créée. 
Celle-ci offre un atout certain afin de se concentrer sur les
problématiques du projet.


\section{Fonctionnement}
\subsection{Algorithme général}
Nous avons opté pour une gestion récursive de la ligne de commande et
des commandes pipées. 

L'algorithme général est décrit ci-dessous : 
\begin{lstlisting}
Lancement du shell 
Si aucun argument fourni dans argv
  BoucleInfinie
  Demander commande
    Si commande valide
      fork() 
      Si père 
         Attendre son fils
      Si fils 
         Relancer shell avec comme unique argument la commande
  FinBoucle
Sinon si un argument est fourni dans argv
  Compter nombre de pipes dans l'argument
  Si pas de pipe
    Executer la commande
  Sinon si une pipe au moins
     pipe()
     fork()
     Si pere
       Executer la commande la plus a droite
     Sinon fils
       Relancer le shell avec comme unique argument l'entree
       utilisateur privee de la derniere commande
     FinSi
  FinSi
FinSi
\end{lstlisting}

Cette gestion récursive oblige une exécution de droite à gauche 
de la ligne de commande. En effet, si l'exécution des commandes se faisait classiquement de gauche à droite, le premier processus une fois fini signalerai au parent qu'il
pourrait continuer alors que toute la chaîne de pipeline n'est pas
forcément finie. Ici, c'est seulement une fois le dernier processus
fini que le père reçoit un signal comme quoi la séquence est finie.


Visuellement, cela donne quelque chose comme sur la figure \ref{pipe} 
\begin{figure}[!h]
\centering
  \includegraphics[scale=0.5]{img/pipes2}
  \caption{Exemple de gestion des pipes sur la séquence ps | sort | less}
\label{pipe}
\end{figure}


\subsection{Redirections}
Dans la version du code actuel, une contrainte sur la redirection est que le nom du fichier doit
suivre \emph{sans espace} le symbole de redirection. On pourrait
au besoin changer le code afin de supporter la présence d'un espace.

\subsection{Changement du répertoire courant}
Le changement de répertoire se fait grace à la détection de la
chaîne \emph{cd} dans la ligne de de commande. On va ensuiter appeler 
\emph{chdir} sur le reste de la chaîne afin de changer de répertoire.

Etant donné que le programme se relance lui même via arg[0], si le
shell a été lancé de façon relative au répertoire courant, un
changement de répertoire va empecher le shell de marcher. Il est donc
nécessaire de le lancer avec un chemin absolu.

Cette problématique ne se poserait pas si le binaire était dans le path.

\section{Vérification de la cohérence de l'entrée utilisateur}

Afin de vérifier la cohérence de l'entrée utilisateur, nous avons mis
en place un petit analyseur avec Flex et Bison.

De cette façon, on évite par exemple que l'utilisateur puisse entrer une commande 
dont la sortie serait redirigée et qui serait en même temps pipée vers
une autre commande.

La grammaire est la suivante :
\begin{lstlisting}
  prog: line;
line : cmd_solo | pipe_line |;

cmd_solo: cmd | cmd_in | cmd_out | cmd_inout;

cmd_begin_pipe : cmd | cmd_in;
cmd_end_pipe : cmd | cmd_out;

cmd_in : cmd IN WORD 
;

out_symbol : OUT | OUT_APPEND;
cmd_inout : cmd out_symbol WORD IN WORD
          | cmd IN WORD out_symbol  WORD 
;

cmd_out : cmd out_symbol WORD 
;

cmd : WORD args 
;

arg_word : WORD | QWORD;

args: args arg_word   | {$$="";}
;

pipe_line : cmd_begin_pipe pipe_cmd  ;
pipe_cmd : PIPE cmd pipe_cmd  | PIPE cmd_end_pipe ;
\end{lstlisting}

et le fichier du lexer :

\begin{lstlisting}
  [-\}$\{a-zA-Z0-9_\./]+ { yylval.str = strdup(yytext); return WORD;}

\"([^\\\"]|\\.)*\" {  yylval.str = strdup(yytext); return QWORD;}
\'([^\\\']|\\.)*\' {  yylval.str = strdup(yytext); return QWORD;}

"|" {return PIPE;}
"<" {return IN;}
">" {return OUT;}
">>" {return OUT_APPEND;}

"\n" { return *yytext; } 
\end{lstlisting}
En l'état actuel, il sert juste à accepter ou refuser une séquence de
commande. On pourrait l'utiliser pour remplir des structures afin de 
disposer de la ligne de commande déjà parsée sans avoir à faire appel
à la fonction tokenize. Cette évolution n'a pas été réalisée.
\section{Améliorations et ajouts possibles}
Il existe encore de nombreuses améliorations possibles à notre projet
\begin{itemize}
  \item Lancement séquentiel des commandes avec ;
  \item Lancement conditionnel des commandes avec \& et \&\&
  \item Globing, auto-complétion, historique des commandes
\end{itemize}
\end{document}
